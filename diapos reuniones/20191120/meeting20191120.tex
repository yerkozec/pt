\pdfoptionpdfminorversion=7
\documentclass{beamer}

\mode<presentation>
{
  \usetheme{Madrid}       % or try default, Darmstadt, Warsaw, ...
  \usecolortheme{default} % or try albatross, beaver, crane, ...
  \usefonttheme{serif}    % or try default, structurebold, ...
  \setbeamertemplate{navigation symbols}{}
  \setbeamertemplate{caption}[numbered]
}

\usepackage{amsmath}
\usepackage[utf8x]{inputenc}
\usepackage{listings}
\usepackage{graphicx}
\usepackage{lmodern}
\usepackage{tikz}

\usetheme{default}

\title{Proyecto de titulo I}
\author{Yerko Zec}
\institute[]{FI - UNAB}
\date{\today}


\begin{document}

\begin{frame}[plain]
  \titlepage
\end{frame}

\addtocounter{framenumber}{-1}

\begin{frame}{Programación detección cara}
\begin{itemize}
    \item Se realizo una función la cual detectaba los vectores que componen la figura y los divide en un arreglo nuevo que consiste en los vectores de cada cara de la siguiente manera.
    $V_i{}_j{}_k{}_l$
    \item Una vez detectadas y divididas las caras dentro del arreglo, se procede a generar un centroide de la cara, el cual se utilizara para la detección por cual cara entro el punto a estudiar.
    
\end{itemize}
\end{frame}

\begin{frame}
 \begin{itemize}
  \item La forma en la que se busca detectar por cual cara entro el punto de prueba, se analizara con respecto al centroide.
  \item Obteniendo la distancia al centroide de la cara se asumirá que el punto entro por esa cara.
 \end{itemize}
\end{frame}


\begin{frame}{Problemas}
  \begin{itemize}
   \item El problema más recurrente es que se busco optimizar el código desde un principio lo cual atraso un tiempo el trabajo.
   \item Falta parte del código que calcula la distancia al punto, porque falta un paso en la generación del centroide.
  \end{itemize}
\end{frame}

\begin{frame}{ToDo}
  \begin{itemize}
   \item Estudiar nuevas maneras de programar el código para buscar optimización.
   \item Con el set de prueba verificar si se detecta correctamente por donde entro el punto estudiado.
  \end{itemize}
\end{frame}


\medskip
\bibliographystyle{plain}
\bibliography{/home/yerkozec/Desktop/pt/memoria/Referencia}

\end{document}
